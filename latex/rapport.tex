\documentclass[a4paper,11pt]{article}

\usepackage[utf8]{inputenc}    % Pour que LaTeX comprenne les accents.
\usepackage{times}             % Police de caractères
\usepackage[english]{babel}    % Traitement du texte adapté aux règles typographiques
                               % de la langue donnée en option (e.g., pour l'espacement
                               % après les ponctuations
\usepackage[T1]{fontenc}
\usepackage{amsmath, amsthm, amssymb}
\usepackage{dsfont}            % Pour les indicatrices
\usepackage{graphicx}
\usepackage{textcomp}
\usepackage{enumerate}
\usepackage{authblk}
\usepackage[colorlinks=true, citecolor=blue, urlcolor=magenta]{hyperref}
                               % Pour créer des hyperliens
\sloppy                        % Ne pas faire déborder les lignes dans la marge


\author{Marc Heinrich \hspace{20mm} Baptiste Lefebvre}
\title{MPRI 2.19 - Programming project \\ Analysis of cyclic attractors for \\ asynchronous boolean models of cellular networks}
\date{February 17, 2015}



\begin{document}

\maketitle


\section{Goal}

% Study one particular model, experiment on it. 
% Several different things were looked at

% Mechanical Reproduction of the experiement (compute the smallest perturbation
% to reach a given state)

% Computation of attractors : two different methods. One relying on MDDs,
% implementing the algorithm from Garg et Al.
% An other that use prime implicants and Integer linear programming to find the attractors.

The goal of this project is to tackle the problem of the identification of
cyclic attractors for asynchronous boolean models of cellular networks, and to
characterize their robustness. In this respect, we used the model published by
\cite{Bonzanni} as a test case. The work we have done spreads in two different
directions. The first one is to try to reproduce the experiment that is
described on the article, to compute the \emph{smallest perturbation} needed to
reach a given state. The second direction is the computation of the
\emph{attractors} of a model. Several different algorithms are already known for
this problem. We implemented two of them and integrated them into GINsim. The
first algorithm relies on \emph{Multivalued Decision Diagrams} (MDDs). The
second one on \emph{prime implicants} and \emph{(0-1) linear programming} to
compute the \emph{symbolic steady states} of a given boolean network.

Our code can be found on GitHub at
\href{https://github.com/biobioinfo/project}{https://github.com/biobioinfo/project}.


\section{What was done}

% Our contribution is an integration into GINsim of several functionnalities to
% solve the problems mentionned above.
% They were added using the plugin mechanism. 

% Compute the smallest perturbation
% Very simple method : compute the set of states that are backward reachable
% from one state, ennumerate them, and keep the closest to the source.

% Computing attractors using MDDs. The algorithm that we used is described in
% garg et Al. 
% Note that an important part of the implementation was not the algorithm
% itself, but some simple operations on Mdds that were not already present into
% MDDlib.
% Big problem understanding the implementation, and extending it four our
% purposes.
% Gives results for regulatory graphs of up to 20 nodes in the synchronous case
% (max 10 asynchronous). 

% ILP based solution ... ? see

Our contribution is an integration into GINsim of several functionnalities to
solve the problems mentionned above. These functionnalities were added using
\emph{GINsim's plugin mechanism} which should make it relatively easy to
integrate.

\subsection{Compunting minimal perturbation}
To compute the minimal perturbation needed to reach a given state, we used a
relatively naive approach. It consists in: given the target state compute all
the states that are \emph{backward reachable}, that is the states we can reach
by taking transitions in the reverse from the target state, and then enumerate
them all to find the closest one from the source. One of the problems of this
method is that it may enumerate a large number of states (eventually the whole
state space) if the target state happens to have many ancestors. We tested this
plugin on a 11 node regulatory graph (from \cite{Bonzanni}) and it manages to
find the smallest perturbation in less than 20 seconds.

\subsection{MDD approach to compute attractors}
As we said before, we also implemented the algorithms from \cite{Garg} and
\cite{Zheng}. The major difficulty that we encountered on this part, was on the
manipulation of MDDs. The algorithm itself is relatively simple, however it
requires to be able to already do some relatively basic operations on the MDDs
we manipulate. So we had to implement some of these (for example : variable
substitution, quantification over variables, or other binary operators like
equality, addition...). In general these operations were implemented using a
naive approach and are probably the main issue in the time complexity. We were
able to test the algorithm on several models from GINsim model repository. On
average, we were able to compute the attractors for synchronous transitions on
models of up to 20 nodes in less than 10 seconds. On larger models, the
computation seemed to take really more time (more than several minutes). For the
asynchronous case, the performance are much worse, and it doesn't manage to find
the attractors in reasonnable time for models of more than a dozen of nodes.

\subsection{(0-1)LP approach to compute symbolic steady states}
Finaly we have also implemented the algorithm from \cite{Klarner}. Our
motivation was to push the limits of size (i.e. node numbers) to compute
attractors without enumerating the whole state space but by working directly on
the boolean regulatory network. The idea is to enumerate all the prime
implicants of a boolean regulatory network, which means for each update function
of a gene to enumerate the configurations where a minimum number of valued
inputs lead to a valued output. This part was implemented by the extraction of
some implicants from the MDD representations of the update functions followed by
an iterative filtering process to reach the prime implicants. Then we formulate
the (0-1) problem described in \cite{Klarner} using \emph{!oj Algorithms}
(or \emph{ojAlgo}, see \href{http://ojalgo.org}{http://ojalgo.org}) which is the
one and only free, open source and pure java code offering a (0-1) solver we
were able to found. This (0-1) problem is simply the optimization-based approach
which allow the efficient computation of maximal symbolic steady states. In other
words, the constraints which select the set of updates which are consistent
(corresponding update functions agree on the input values) and stable (output
values equals input values, when defined). We were able to test a non-final
version of the algorithm on models of about ten nodes without significant time
effect. More extensive testing will be presented during the oral presentation.


\section{Possible improvements}
% Compute perturbation :
% If you only want the distance, you should be able to 
% express the problem entirely in terms of Mdds, and avoid the (costly ennumeration part)
% Comutation of attractors : a lot of naïve implementation that could be improved to speed up the process
% Use of lazy evaluation on the mdds to reduce the memory requirements?

From what we have done, several improvements can be made. First, for the problem
of computing the minimum perturbation, it should be possible to solve the
problem using only MDDs. This would have the advantage of not enumerating the
possibly many states as in our current version. However, we would only get this
way the number of genes that must be modified, and not which one.
For the MDD approach to compute attractors, the main solution to speed up the
process would probably be to implement base operations (like variable substitution, universal and existential quantification...) more carefully.
One other idea could be to use the observation that states that do not have any
antecendents are necessarily transients. Since it should be possible to check
this property using MDDs, it should be possible to reduce a great part of the
state space (actually all the transient states in the synchronous case).
Finally for the (0-1) linear programming approach, a similar solution to speed
up the process might be to replace the oj! Algorithms by a dedicated (0-1)
solver which takes advantage of the restricted maximal in-degree and its
implications on the (0-1) problem to use the appropriate heuristic. A simplier
way of enhancement is to optimize the part of the algorithm which enumerate the
prime implicant, in fact this algorithm works on a list of hypothetical prime
implicants which are compared in pairs several times but it could reduce this
list by removing progressively the elements which are certainly prime
implicants. A related improvement would be to find an algorithm which works
only with MDDs, such an algorithm is more interesting but more difficult to
conceive.


% Bibliography

\bibliographystyle{apalike}
% Style among : abbrv, acm, alpha, apalike, ieeetr, plain, siam and unsrt.
\bibliography{rapport.bib}

\end{document}
