\documentclass{beamer}

\usepackage[utf8]{inputenc}
\usepackage[T1]{fontenc}
\usepackage[english]{babel}

\usetheme{default}


\title{MPRI 2.19 - Programming project}
\subtitle{Analysis of cyclic attractors for \\ asynchronous boolean models of cellular networks}
\author{Marc Heinrich \and Baptiste Lefebvre}
\institute{École Normale Supérieure, Computer Science Department}
\date{February 24, 2015}


% README
% 
% Soutenance mardi 24 février entre 09h30 et 11h30.
% 
% Objectif:
%   - présenter le projet de programmation
%   - présenter qui a fait quoi pour la réalisation du projet
%   - présenter les résultats obtenus
%
% Durée:
%   - 25 minutes de présentation
%   - 15 minutes de questions


\begin{document}


\section*{Title}

\begin{frame}
  \titlepage
\end{frame}


\section*{Table of contents}

\begin{frame}
  \frametitle{Table of contents}
  \tableofcontents
\end{frame} 


% Presentation %%%%%%%%%%%%%%%%%%%%%%%%%%%%%%%%%%%%%%%%%%%%%%%%%%%%%%%%%%%%%%%%%

\section{Presentation}

\begin{frame}
  \frametitle{Presentation}    
  %Implementation of algorthms integrated in GINsim
  TODO: complete
  
  TEST: cite \cite{Bonzanni}
\end{frame}

\subsection{Introduction to MDDs}
%Maybe not put here
%Take slides from previous presentation
\begin{frame}
% 1-2 minutes
\frametitle{BDDs/MDDs}
	TODO: quick description of the data structure
\end{frame}

\begin{frame}
TODO: quick use case (set/boolean function representation) and operations available
\end{frame}

% Distribution of tasks %%%%%%%%%%%%%%%%%%%%%%%%%%%%%%%%%%%%%%%%%%%%%%%%%%%%%%%%

\section{Distribution of tasks}

\begin{frame}
% 1 minute
  \frametitle{Distribution of tasks}
  TODO: complete
- Finding minimum perturbation
- MDD based algorithm for attractors
- LIP based algorithm  
  
\end{frame}

\begin{frame}
% 2- minutes
	\frametitle{Finding the minimum perturbation}
	TODO: explain problem
	TODO: explain example Bonzanni
	TODO: explain simple solution
	TODO: idea of improvement
\end{frame}

\begin{frame}
% 3-4 minutes
	\frametitle{Finding attractors}
	TODO: describe Garg Algorithm (pseudo-code)
	TODO: synchrone and asynchrone
	TODO: idea of improvement
\end{frame}

\begin{frame}
% Bapt : LIP based algo
	TODO 

\end{frame}


% Results %%%%%%%%%%%%%%%%%%%%%%%%%%%%%%%%%%%%%%%%%%%%%%%%%%%%%%%%%%%%%%%%%%%%%%

\section{Results}

\begin{frame}
  \frametitle{Results}
  %Demonstration on GINsim of the different plugins?   
  TODO: complete
\end{frame}


\section*{References}

\begin{frame}
  \frametitle{References}
  \bibliographystyle{apalike}
  \bibliography{presentation.bib}
\end{frame}


\end{document}
